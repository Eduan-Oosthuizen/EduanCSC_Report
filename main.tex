\documentclass[12pt, a4paper]{report}

\usepackage{}

\begin{document}
\section{Introduction}

\section{Information visualisation}
As noted by Few (2009: 11) the term \emph{data analysis} is one so often used that its meaning has become unclear./
To analyse means to break a system into its constituting elements with the goal of gaining understanding.
\emph{Data analysis} is loosely applied in many fields to include the preparation of data for analysis, and then comparing components
of the data and the relative relationships between these components for the purpose of understanding the information conveyed by the data (Few, 2009: 11).

\emph{Visualisation}, as relevant here, refers to graphical representation information whilst
 \emph{data visualisation} allows for its exploration, examination and communication (Few 1009, 12)./
 The relevant term when a time-series data analysis system is considered would be \emph{information visualisation}
 which is properly defined by Card, Mackinlay and Shneiderman as "the use of computer supported,
 interactive, visual representations of abstract data to amplify cognition," (Card, Mackinlay & Shneiderman, 1999 : 7)./
 Few (2009: 13) elaborates on each of the elements included in this definition.
 In each of these elaborations ideas are conveyed that serve as support for the development of a system such as Evert and 
 capture some of its key envisioned capabilities and are therefore repeated here.
 	\emph{Computer-supported}: visualisations are computer generated and usually viewed on a computer screen.
	\emph{Interactive}: the visualisations can be manipulated in a natural, responsive and direct manner.
	\emph{Visual representations}: the visual from in which data is displayed exploits properties such as location,
        shape, length, color and size to form a picture that compliments the recognition of trends that might not be otherwise
        visible.
	\emph{Abstract data}: information such as quantitative data, processes or relationships are considered abstract as
        such information has no natural form.
        Visualisations are therefore tasked with using visual representations to represent the data in meaningful ways.
	\emph{Amplify cognition}: our ability to think about the represented information is extended
        due to these representations being easily comprehended by our brains.
	
 As Few (2009, 2, 13) observes, the purpose of information visualisation is to help humans think in order to make well informed decisions./
 If the focus is then kept solely on this purpose it becomes apparent that the interaction with a computer system
 that facilitates the process of data analysis should,
 ideally fade from the cognitive load so that the full human capacity may be applied solely to understanding the information
 and its implications.
 Card (2004) echoes this purpose by stating that information visualisation is about external cognition- hoe resources outside the mind
 boost the cognitive capabilities of the mind.
 
 Tukey and Wilk (1965) made the following observation:
 `` As in any other science, what is doine in data analysis is very much a product of each day's technology.
 Every external technological development of major relevance- organised tables of functions, knowledge of the mathematical consequences of
 the Gausian law of erro, desk calculators, stored-program electronic computers, graphical display facilities-
 has been accompanied by a tendency to rediscover the importance and identity of data analysis.'

 Why information visualisation remains relevant to engineering diciplines is well stated by Ware (2004, xxi)./
 He describes the human visual system as a most effective pattern seeker and that the eyes and visual cortex of the
 brain act as a parrallel processor to provide the highest bandwidth channel into human cognitive centres.

 There is therefore a need to employ modern computing power and development to make data analysis easier and more
 effective (Few: 2009, 17).

 \section{Design of time-series data analysis system}
 In the design of a computer-based system it is necessary to apply good design principles that allow for ease of use that compliments the
 functionality offered by the system. If this is not the case some of the cognitive load on humans that has been removed by the system's tools
 is again taken up in trying to interact with an unaccomodating interface.

 \section{Design of visualisations that compliment human perception}
 Few (2009: 33) explores the use of knowledge concerning human perception to ensure effective visualisation of information.
 As an introduction to the following discussion it must be noted that human visual perception:
 % Enter unordered list here, use dashes?
   \item{is drawn to contrasts}
   \item{drawn to identify familiar patterns}
   \item{has very limited working memory}
     
     Humans percieve several basic attributes of a visual image before the image is consciously addressed. These attributes are termed
     \emph{pre-attentive attributes}(Bertin, 1963; Few: 2009, 38).

     With abstract information, such as time series-data, pre-attentive attirbutes are exploited visualisation as they determine what
     visual details are passed to conscious attention (Ware: 2004, 163). FIGURE?? shows these attributes as reported by Ware (2013, WEB)
     %
     %
     %
     %
     %
     %
     %
     %

     Each of these attributes are used to accentuate different properties information./
     Shape and colour are principally used for distinction of objects in a visualisation. These pre-attentive attributes naturally
     distinguish sets of information that are qualititatively different from each other. (Few: 2009, 40)/ %QUESTION: inline referencing for a paragraph?
     There are also pre-attentive attributes that are naturally percieved quantitatively without arbitrary values being assigned to them, those
     of which identified by Few (2009: 41) are included in FIGURE??. Of these, only length and 2D position are percieved with a high degree of precision.
     %
     %
     %
     %
     %
     %
     %
     %
     %

     It is also worthy to note that 
     
 
\end{document}

Ware (2004, xxi) states that the human visual system can only identify patterns when they are presented effectively according to the
rules of human visual perception. This motivates great care to be taken in inventing information visualisation to adhere to how humans
see to allow for the purpose of information visualisation to help humans think (Few: 2009, 2, 13) to be met.
